There exists many places in society where the degree of human occupancy and movement flow is desirable to know as basis for decision making. Such data answers if it is necessary to build more rooms and provides knowledge of actual user or consumer patterns. Example usages are measuring reasource usage of public spaces, or which part of a store that attracts most people. It provides vast opportunities in resource management, marketing, sales and scheduling. There exist some plausible solutions to estimating the number of people at a location such as using cell phones or motion detectors, but this project aims at an image based approach with the possible benefits of being both cheaper and more robust.

\subsection{Background}
Today Linköping University has many places with similar functionality, e.g. student kitchens where students are provided with the ability to warm food brought with them. Linköping University has several such kitchens all over its campuses. Critics claim that there are too few student kitchens with microwave ovens and that the existing ones usually are overcrowded. That all kitchens are overcrowded at the same time has not been confirmed by sample inspections. One standing hypothesis is that students don't know where all the kitchens are nor that they want to risk going to a kitchen in another building in case that is full as well.\\
\\
Linköping University has an ongoing project with the purpose of enabaling the students to see the usage of some of the schools resources (e.g. group rooms) online. The aim of this project is to supply that system with data regarding the usage of student kitchens. It will provide all students with the ability of visualising the crowdedness of each kitchen, thus providing them with the means of finding the closest, least occupied kitchen available.

\subsection{Involved Parties}
Three parties are involved:
\begin{itemize}
\item Liu IT, the Division for IT servidces at Linköping University.
\item Computer Vision Laboratory, Department of Electrical Engineering, Linköping University.
\item A group av students taking the course TSBB11 2013, listed in the \textit{Participants} table, page (ii).
\item The students at Linköping University.
\end{itemize}

\subsubsection{Customer}
Liu IT, represented by Joakim Nejdeby, CIO at Linköping University.

\subsubsection{Supervisor}
Ph.D Fahad Khan at the Computer Vision Laboratory, Department of Electrical Engineering, Linköping University.

\subsubsection{End users}
Students at Linköping University that want to use the student kitchens.

\subsection{About this document}
This document contain the requirements of the project. It is divided into different modules or aspects, each with further subdivisions, all containing explanatory text and functional requirements. Each functional requirement is placed in a table of the form showed below.

\subsubsection{Requirement priorities}
Each requirement has three different types, the meaning of each one is presented below:

\begin{enumerate}
	\item type one constitutes a mandatory requirement, meaning this feature has to be fulfilled by at the time specified ion the description. If no time is specified, the requirement has to be fulfilled by the time of the final delivery (see section \ref{sec:delivery}). %Länk till sektionen med leveranskraven
	\item A requirement with type two is a requirement to be met if extra time is available.
	\item A type three requirement is more of a suggestion on how to improve the system even further after the final delivery.
\end{enumerate}
