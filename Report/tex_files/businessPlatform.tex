Bla bla bla

\subsection{Idé}

\subsection{Produkt}

\subsection{Marknad}


\subsection{Organisation}

\subsection{Erfarenhet}

\subsection{Engagemang}

\subsection{Relation till kunder}

\subsection{Relation till andra företag}
Det lades mycket fokus på att knyta kontakter med företag på andra marknader. Det skrevs avtal med bland annat Facebook och Coca-Cola. Detta gjorde Spotify konkurrenskraftiga gentemot andra musiktjänster och gav dem även möjlighet att växa sig större snabbare. Genom att synas tillsammans med stora etablerade företag i diverse reklamkampanjer och liknande kunde även Spotifys eget varumärke förstärkas. 

Samarbetet med Facebook kan anses vara av extra stor vikt. Möjligheterna att dela med sig av sin musik och se vad andra lyssnar på blev helt plötsligt enorma. I samband med att Spotify presenterades på Facebooks F8 konferens sågs till exempel en ökning av månatliga aktiva användare på en miljon. 

\includegraphics[width=\linewidth]{images/MAU2011.png}
\includegraphics[width=\linewidth]{images/DAU2011.png}



Nedan följer en lista med företag Spotify på något vis samarbetat med:

\begin{itemize}
  \item \emph{Telia}, Sverige - 2009. \\
Få sex månader gratis premium. Förde in Spotify i Telias många butiker i Sverige och även på deras hemsida.

\item \emph{SFR}, Frankrike - 2009. \\
Spotify ingår som del i månadskostnaden. Spotify syns på deras hemsida.

\item \emph{Nissan}, Japan - 2009. \\
Nissan Quasquai var Powered By Spotify. Detta visades för alla intresserade kunder på hemsida.

\item \emph{Sonos}, USA	- 2010.\\
Spotifys första hårdvaruintegration. Sonos är en Amerikansk hi-fi-aktör inriktad mot trådlösa musiksystem. Sonos närvarar på över 60 marknader.

\item \emph{Chevrolet}, USA	- 2011.\\
De första 150 000 som började prenumerera på Chevrolet på Facebook fick en Spotify inbjudan. Detta skedde under den tid då inbjudan krävdes för att få tillgång till Spotify.

\item \emph{Onkyo}, Japan - 2011.\\
Japansk hi-fi aktör som var först med att erbjuda Spotify genom deras hemmabioanläggning.

\item \emph{Logitech}, USA - 2011.\\
Spotify infördes i Logitechs Squeezebox, en nätverksmusikenhet. Detta möjliggör för streaming från Spotify och uppspelning av spellistor. Logitech tillverkar datortillbehör och annan musikutrustning.

\item \emph{Facebook}, USA - 2011.\\
Spotify integreras i Facebook och vice-versa. Man kan lyssna på musik direkt i Facebook, dela spellistor och se vad kompisar lyssnar på.

\item \emph{Shazam}, Storbrittanien - 2011.\\
En tjänst lik sound hound som "lyssnar" på en låt och säger vad den heter. Detta kopplades ihop med Spotify så att låten direkt kan spelas via deras bibliotek.

\item \emph{SEAT}, Spanien - 2011. \\
Få en Samsung mobil med Spotify installerat gratis i sex månader. Reklam syntes på TV, billboards, youtube, tidningar, hemsida.

\item \emph{Coca-Cola}, USA	- 2012.\\
Sammarbete med Coca-Cola can vara mycket lönsamt då de är aktiva i cirka 200 länder vilket för ut Spotify på nya marknader. Samtidigt kan Spotify bistå med musik till Coca-Colas online-marknadsföring på till exempel Facebook.

\item \emph{Virgin Mobile}, Storbrittanien - 2012. \\
Få tre månaders gratis premium. Spotify syns på deras hemsida.

\item \emph{Volkswagen}, Tyskland - 2012.\\
Spotify musik spelas på Volkswagens hemsida och folk kan önska musik som ska spelas. Spotifys logo syns tydligt och binder samman de två företagen.
\end{itemize}
